% This is "sig-alternate.tex" V2.0 May 2012
% This file should be compiled with V2.5 of "sig-alternate.cls" May 2012
%
% This example file demonstrates the use of the 'sig-alternate.cls'
% V2.5 LaTeX2e document class file. It is for those submitting
% articles to ACM Conference Proceedings WHO DO NOT WISH TO
% STRICTLY ADHERE TO THE SIGS (PUBS-BOARD-ENDORSED) STYLE.
% The 'sig-alternate.cls' file will produce a similar-looking,
% albeit, 'tighter' paper resulting in, invariably, fewer pages.
%
% ----------------------------------------------------------------------------------------------------------------
% This .tex file (and associated .cls V2.5) produces:
%       1) The Permission Statement
%       2) The Conference (location) Info information
%       3) The Copyright Line with ACM data
%       4) NO page numbers
%
% as against the acm_proc_article-sp.cls file which
% DOES NOT produce 1) thru' 3) above.
%
% Using 'sig-alternate.cls' you have control, however, from within
% the source .tex file, over both the CopyrightYear
% (defaulted to 200X) and the ACM Copyright Data
% (defaulted to X-XXXXX-XX-X/XX/XX).
% e.g.
% \CopyrightYear{2007} will cause 2007 to appear in the copyright line.
% \crdata{0-12345-67-8/90/12} will cause 0-12345-67-8/90/12 to appear in the copyright line.
%
% ---------------------------------------------------------------------------------------------------------------
% This .tex source is an example which *does* use
% the .bib file (from which the .bbl file % is produced).
% REMEMBER HOWEVER: After having produced the .bbl file,
% and prior to final submission, you *NEED* to 'insert'
% your .bbl file into your source .tex file so as to provide
% ONE 'self-contained' source file.
%
% ================= IF YOU HAVE QUESTIONS =======================
% Questions regarding the SIGS styles, SIGS policies and
% procedures, Conferences etc. should be sent to
% Adrienne Griscti (griscti@acm.org)
%
% Technical questions _only_ to
% Gerald Murray (murray@hq.acm.org)
% ===============================================================
%
% For tracking purposes - this is V2.0 - May 2012

\documentclass{template}
\usepackage[utf8x]{inputenc}
%--------------------------------------------------------------------
% Pakete
%--------------------------------------------------------------------

% KOMA-Script
\usepackage[table]{xcolor}          % Erweitertes Farbpaket mit vielen Farbmodellen (notwenidig zur Farbdefinition)
\usepackage{listings}               % Quellcode einbinden und formatieren
\renewcommand{\lstlistlistingname}{Verzeichnis der Listings}
\definecolor{darkgreen}{rgb}{0,.4,0}
\definecolor{darkviolett}{rgb}{.4,0,.4}
\newcommand{\listingswidth}{\textwidth-1cm}
\lstset{
    language=C,                  % oder  C++, Pascal, {[77]Fortran}, ...
    numbers=left,                   % Position der Zeilennummerierung
    firstnumber=auto,               % Erste  Zeilennummer
    basicstyle=\ttfamily\small,     % Textgröße  des Standardtexts
    keywordstyle=\ttfamily\color{darkviolett},    % Formattierung Schlüsselwörter
    commentstyle=\ttfamily\color{darkgreen},       % Formattierung Kommentar
    stringstyle=\ttfamily\color{blue},             % Formattierung Strings
    %keywordstyle=\ttfamily\color{black},
    %commentstyle=\ttfamily\color{black},
    %stringstyle=\ttfamily\color{black},
    numberstyle=\tiny,              % Textgröße der Zeilennummern
    stepnumber=1,                   % Angezeigte Zeilennummern
    numbersep=5pt,                  % Abstand zw. Zeilennummern und Code
    aboveskip=15pt,                 % Abstand oberhalb des Codes
    belowskip=11pt,                 % Abstand unterhalb des Codes
    captionpos=b,                   % Position der Überschrift
    linewidth=15.8cm,               % TODO \textwidth-2em
    xleftmargin=10pt,               % Linke Einrückung
    frame=single,                   % Rahmentyp
    breaklines=true,                % Umbruch langer Zeilen
    showstringspaces=false          % Spezielles Zeichen für Leerzeichen
}

\begin{document}
%
% --- Author Metadata here ---
\conferenceinfo{Hochschule für Telekommunikation Leipzig\\Modul:Protokolle\\WS 2014/15}{}
%\CopyrightYear{2007} % Allows default copyright year (20XX) to be over-ridden - IF NEED BE.
%\crdata{0-12345-67-8/90/01}  % Allows default copyright data (0-89791-88-6/97/05) to be over-ridden - IF NEED BE.
% --- End of Author Metadata ---

\title{UDP-Lite {\ttlit}}
%
% You need the command \numberofauthors to handle the 'placement
% and alignment' of the authors beneath the title.
%
% For aesthetic reasons, we recommend 'three authors at a time'
% i.e. three 'name/affiliation blocks' be placed beneath the title.
%
% NOTE: You are NOT restricted in how many 'rows' of
% "name/affiliations" may appear. We just ask that you restrict
% the number of 'columns' to three.
%
% Because of the available 'opening page real-estate'
% we ask you to refrain from putting more than six authors
% (two rows with three columns) beneath the article title.
% More than six makes the first-page appear very cluttered indeed.
%
% Use the \alignauthor commands to handle the names
% and affiliations for an 'aesthetic maximum' of six authors.
% Add names, affiliations, addresses for
% the seventh etc. author(s) as the argument for the
% \additionalauthors command.
% These 'additional authors' will be output/set for you
% without further effort on your part as the last section in
% the body of your article BEFORE References or any Appendices.

\numberofauthors{1} %  in this sample file, there are a *total*
% of EIGHT authors. SIX appear on the 'first-page' (for formatting
% reasons) and the remaining two appear in the \additionalauthors section.
%
\author{
% You can go ahead and credit any number of authors here,
% e.g. one 'row of three' or two rows (consisting of one row of three
% and a second row of one, two or three).
%
% The command \alignauthor (no curly braces needed) should
% precede each author name, affiliation/snail-mail address and
% e-mail address. Additionally, tag each line of
% affiliation/address with \affaddr, and tag the
% e-mail address with \email.
%
% 1st. author
\alignauthor
Johannes Hamfler\\
       \affaddr{KMI12}\\
       \email{johannes.hamfler@hft-leipzig.de}
% 2nd. author
%\alignauthor
%2nd author
%       \affaddr{affiliation}\\
%       \email{e-mail}
% 3rd. author
% \alignauthor Lars Th{\o}rv{\"a}ld\titlenote{This author is the
% one who did all the really hard work.}\\
%        \affaddr{The Th{\o}rv{\"a}ld Group}\\
%        \affaddr{1 Th{\o}rv{\"a}ld Circle}\\
%        \affaddr{Hekla, Iceland}\\
%        \email{larst@affiliation.org}
% \and  % use '\and' if you need 'another row' of author names
% % 4th. author
% \alignauthor Lawrence P. Leipuner\\
%        \affaddr{Brookhaven Laboratories}\\
%        \affaddr{Brookhaven National Lab}\\
%        \affaddr{P.O. Box 5000}\\
%        \email{lleipuner@researchlabs.org}
% % 5th. author
% \alignauthor Sean Fogarty\\
%        \affaddr{NASA Ames Research Center}\\
%        \affaddr{Moffett Field}\\
%        \affaddr{California 94035}\\
%        \email{fogartys@amesres.org}
% % 6th. author
% \alignauthor Charles Palmer\\
%        \affaddr{Palmer Research Laboratories}\\
%        \affaddr{8600 Datapoint Drive}\\
%        \affaddr{San Antonio, Texas 78229}\\
%        \email{cpalmer@prl.com}
}
% There's nothing stopping you putting the seventh, eighth, etc.
% author on the opening page (as the 'third row') but we ask,
% for aesthetic reasons that you place these 'additional authors'
% in the \additional authors block, viz.
%\additionalauthors{Additional authors: John Smith (The Th{\o}rv{\"a}ld Group,
%email: {\texttt{jsmith@affiliation.org}}) and Julius P.~Kumquat
%(The Kumquat Consortium, email: {\texttt{jpkumquat@consortium.net}}).}
%\date{13 Dezember 2014}
% Just remember to make sure that the TOTAL number of authors
% is the number that will appear on the first page PLUS the
% number that will appear in the \additionalauthors section.

\maketitle
\begin{abstract}
In diesem Dokument wird das Leightweight User Datagram Protokoll (UDP-Lite) beschrieben,
welches ähnlich UDP ist. Der Focus dieses Dokuments liegt in der Beschreibung der Vorteile,
die UDP-Lite gegenüber UDP aufweisen kann. Des Weiteren wird das Protokoll in das
ISO OSI-Referenzmodell eingeordnet und die Auswirkungen auf andere Schichten in diesem beschrieben.
Dieses Dokument orientiert sich stark am RFC.

% \LaTeX\ {\em alternate} \textit{bla}
\end{abstract}

% A category with the (minimum) three required fields
%\category{H.4}{Information Systems Applications}{Miscellaneous}
\category{Ausarbeitung}{Protokolle}{RFC-Protokolle}
%A category including the fourth, optional field follows...
%\category{D.2.8}{Software Engineering}{Metrics}[complexity measures, performance measures]

\terms{Beschreibung von UDP-Lite}

\keywords{UDP-Lite}

\section{Einleitung}

UDP, welches im RFC 768 beschrieben ist, wird seit Jahren als verbindungsloses Protokoll verwendet
und ist weit verbreitet. Auch heute hat dieses große Bedeutung für Sprachdienste,
Videokommunikation und Echtzeitübertragung. Der Vorteil des Protokolls gegenüber TCP
liegt vor allem bei der Sprachkommunikation darin, dass verlorene und fehlerhafte Datenpakete
nicht erneut übertragen werden, da diese nach wenigen Millisekunden schon nicht mehr von Bedeutung sind.\\

In RFC 3828 findet sich die Beschreibung des UDP-Lite Protokolls.
UDP-Lite versucht das Problem der fehlerhaften Pakete, welche beim Empfänger gelöscht werden,
zu mindern, indem die Option besteht fehlerhafte Pakete dennoch zu verwenden
und an höhere Schichten weiterleiten zu können. Bei Sprachdiensten hätte dies den Vorteil,
dass der in einer höheren Schicht angesiedelte Codec die korrekten Bits
auf eine bestimmte Weise verarbeitet, so dass diese nützlich für die Anwendung sind.
Fehlerhafte Bits könnten für den Codec ebenfalls einen Nutzen darstellen,
so dass UDP-Lite in diesem Zusammenhang einen Vorteil darstellen würde.


%\textit{proceedings}
%(18 $\times$ 23.5 cm [7" $\times$ 9.25"])
%\footnote{Two of these, the {\texttt{\char'134 numberofauthors}}}
%{\texttt{\char'134 alignauthor}}
%{\texttt{\char'134 balancecolumns}}

%#############################################


\section{Protokollspezifikation}

\subsection{Einordnung in das OSI-Referenzmodell}

UDP-Lite befindet sich auf Schicht 4, der Transportschicht des OSI-Referenzmodells.
Trotz dass UDP-Lite für eine volle Funktionalität und die Verwendung der
Stärken des Protokolls auf andere Schichten angewiesen ist, stellt es eine
hohe Kompatibilität mit UDP bereit und befindet sich deshalb wie UDP auf Schicht 4.

% {\secit UDP-Lite}
\subsection{Einteilung des Payloads}

Da manche Codecs die Fähigkeit besitzen 
beschädigten Payload zu behandeln und nützliche Informationen
aus diesem zu extrahieren, wurde bei UDP-Lite ein Paket in
zwei Teile aufgegliedert werden. Ein Teil kann mit einem Fehlerkorrekturwert
überprüft werden, um die Integrität der darin enthaltenen Daten zu sichern,
ein anderer Teil kann ohne Prüfsumme vorhanden sein.\\

In dem Teil, in welchem
eine Fehlerüberprüfung stattfinden soll, werden üblicherweise Steuerinformationen
übertragen, welche unbedingt fehlerfrei vorhanden sein müssen, 
um die Parameter des Payloads beim Empfänger richtig interpretieren zu können.
Sollte der Payload in diesem Teil beschädigt sein, wird das Paket beim Empfänger in der
Transportschicht verworfen.\\

Der andere Teil des Payloads, welcher beim klassischen UDP üblicherweise Daten enthält,
welche nicht zwingend neu übertragen werden müssen, kann ohne Fehlerkorrektur übertragen werden,
damit die darüber liegenden Schichten auch beschädigte Daten bearbeiten können,
um aus diesen ebenfalls nützliche Informationen für eine Anwendung zu extrahieren.
Da in diesem Teil nicht überprüft wird ob Fehler vorhanden sind,
wird der Payload nicht für die Entscheidung der Weiterleitung an höhere Schichten verwendet.\\

Wird eine Prüfsumme über das gesamte Paket angewandt, so ist UDP-Lite zwar semantisch identisch
zu UDP, wird jedoch unterschiedlich beim Empfänger behandelt.\\

\subsection{Beobachtungen aus dem RFC}
Im RFC wurden einige Beobachtungen zu der Datenübertragung erläutert, welche hier kurz erwähnt werden.\\

Es wurden folgende Codecs als Beispiele genannt, welche mit UDP-Lite
eine Verbesserung der decodierten Daten erreichen können:
\begin{itemize}
\item AMR speech codec [RFC-3267]
\item Internet Low Bit Rate Codec [ILBRC]
\item error resilient H.263+ [ITU-H.263]
\item H.264 [ITU-H.264; H.264]
\item MPEG-4 [ISO-14496] video codecs)
\end{itemize}

Des Weiteren ist es nützlich, wenn niedrigere Schichten beschädigte IP Pakete
weiterleiten, wenn dies verlangt wird. Sollten Verbindungen sich ihrer
Fehleranfälligkeit bewusst sein, so ist es möglich, dass eine physische Verbindung
eine höhere Sicherheit für sensible Daten gewährleistet, was durch verschiedene
Fehlerkorrekturverfahren erreicht werden kann.\\

Außerdem sollte die Transport- und Vermittlungsschicht höher gelegene
Applikationen nicht an ihrer Ausführung hindern, weil Pakete beschädigt sind.
UDP eigent sich deshalb nur bedingt, da bei diesem die Prüfsumme gesetzt
sein muss. Bei IP ist dies nicht der Fall.\\

\subsection{Der UDP-Lite Header}

Nachfolgend ist der UDP-Header abgebildet.

\begin{lstlisting}[linewidth=0.47\textwidth]
     0              15 16             31
    +--------+--------+--------+--------+
    |     Source      |   Destination   |
    |      Port       |      Port       |
    +--------+--------+--------+--------+
    |    Checksum     |                 |
    |    Coverage     |    Checksum     |
    +--------+--------+--------+--------+
    |                                   |
    :              Payload              :
    |                                   |
    +-----------------------------------+
\end{lstlisting}
\cite{rfc:udplite}

Dieser unterscheidet sich von dem UDP-Header nur in der Hinsicht, dass
das Length-Feld mit einem Cecksum-Coverage-Feld ausgestattet wurde.
Dieses ist dazu da, um die Länge anzugeben, bis wohin die Prüfsumme
berechnet wird. Dies war möglich, da die Information über die Länge
des Pakets aus IP-Paketen entnommen werden kann.\\

\subsection{Beschreibung der Felder der PDU}

Das Source- und Destination-Port-Feld sind dem von UDP gleich, wobei
das Checksum-Coverage-Feld die Länge in Bit-Oketetten angibt, die von
der Prüfsumme einbezogen werden. Hierbei wird ab dem ersten Oktett
mit dem Zählen angefangen.\\

Der Header muss dabei immer mit einer Prüfsumme gesichert werden.
Eine Prüfsumme von 0 bedeutet dabei, dass das gesamte Paket in die
Prüfsumme einbezogen wird. Die Prüfsumme des UDP-Headers muss 0 oder
mindestens 8 sein, was bedeutet, dass die 8 Bytes des Headers beinhaltet
sein müssen. Ein Paket mit einer Prüfsummenlänge zwischen 1 und 7 muss
beim Empfänger verworfen werden, da dieses die Bedingung der Sicherung des Headers nicht erfüllt.
Das berechnete Prüfsummenfeld muss den Pseudo-Header von IP enthalten.
Zusätzlich müssen UDP-Lite Pakete, die eine größere Prüfsummenlänge als
der IP-Header annehmen, ebenfalls verworfen werden. Wäre dies nicht der Fall,
könnte die Länge des UDP-Lite-Pakets nicht festgestellt werden.\\

Da das Prüfsummenfeld ein 16-Bit-Komplement der Summe
des Einerkomplements des Pseudo-Headers darstellt, muss das UDP-Lite-Paket
im Payload ein vielfaches von 2Byte aufweisen. Notfalls wird dieses mit
Nullen aufgefüllt. Die Informationen, aus welchen die Prüfsumme berechnet
wird, werden dem IP-Header entnommen. Bevor die Prüfsumme berechnet wird,
muss das Prüfsummenfeld jedoch auf 0 gesetzt werden. Sollte nach der
Berechnung die Prüfsumme 0 ergeben, so werden 16 Einsen übertragen.\\

Da manche Anwendungen die UDP-Lite benutzen möglicherweise keine
Fehlerbehandlung wünschen, kann hier die Prüfsummenlänge einfach
auf 8 gesetzt werden, um den Payload nicht mit einbeziehen zu müssen.
Dadurch wird sichergestellt, dass die Steuerinformationen in jedem
Fall korrekt ankommen.\\


\subsection{Der Pseudo-Header}

Der Pseudo-Header von UDP-Lite unterscheidet sich von dem in UDP
insofern, dass der Wert des Längenfeldes nicht vom UDP-Lite-Header genommen wird,
sondern von Informationen aus den IP-Paketen. Dabei wird die
Berechnung gleich wie bei TCP ausgeführt, was bedeutet, dass
nicht nur der Payload, sondern auch der Header von UDP-Lite mit
einbezogen wird. Dadurch, dass die Länge auf diese Weise berechnet,
muss das Prüfsummenfeld bei 8 beginnen und ermöglicht so einen
einfachen softwareseitigen Vergleich.\\


\subsection{Die Anwendungsschnittstelle}

Die Anwendungsschnittstelle stellt die gleichen Funktionen wie
bei UDP zur Verfügung. Des Weiteren sollte eine Möglichkeit der
sendenden Anwendung bestehen, den Prüfsummenlängenwert an UDP-Lite
zu übertragen. Zumindest sollte jedoch die Möglichkeit für eine
empfangende Anwendung bestehen, die Weiterleitung von Paketen mit
Prüfsummenlängen kleiner als eine Festgelegte zu blockieren.\\

Im RFC wird empfohlen, dass UDP-Lite standardmäßig das Verhalten
von UDP imitieren sollte, indem das Prüfsummenlängenfeld der
Länge des UDP-Lite-Pakets entsprechen soll, um das gesamte Paket
verifizieren zu können. Über einen expliziten Aufruf, einem
sogenannten System Call, beim Sender, sollen Anwendungen die 
fehlertolerant sind UDP-Lite ihre Fehlertoleranz
mitteilen. Eine empfangende Anwendung die ebenfalls eine teilweise
angewandte Prüfsumme nutzen will, sollte dies ebenfalls über
einen solchen Aufruf kund geben.\\

Da im Internet Pfade variieren und verschiedene Eigenschaften aufweisen,
können keine pauschalen Aussagen über die Fehlerschemas einer
Verbindung gemacht werden. Deshalb sollten Anwendungen die UDP-Lite
nutzen keine Annahmen der Fehler in einem UDP-Lite-Paket machen,
solange der Bereich nicht in die Berechnung der Prüfsumme
mit einbezogen wurde. Anwendungen sollten deshalb wenn nötig
ihre eigenen Fehlerprüfmechanismen nutzen.\\


\subsection{Die IP-Schnittstelle}

Wie bei UDP muss auch UDP-Lite den Pseudo-Header der 
IP-Implementierung erhalten, welcher die IP-Adresse 
und Protokollfelder des IP-Headers beinhaltet, sowie die
Länge des IP-Payloads, welche aus dem Längenfeld der IP-Pakete
entnommen werden kann.
Der Sender darf dabei jedoch nicht den IP-Payload mit Padding-Bytes
auffüllen, da die Länge des UDP-Lite-Pakets daraus entnommen werden soll.\\


\subsection{IP-Jumbo-PDUs}

Da das Prüfsummenlängenfeld Werte bis 65535 annehmen kann,
können genaue Prüfsummenlängen benutzt werden. Dies ist bei
Jumbo-PDUs (Jumbogrammen) nicht der Fall. Es kann entweder das
gesamte Paket mit der Prüfsummenlänge von 0 oder alle Oktette bis
zum 65535ten Oktett einschließen.\\


\subsection{Betrachtung der niedrigeren Schichten}

Frames, die UDP-Lite-Pakete enthalten dürfen von niedrigeren
Schichten nicht verworfen werden, da die Fehlerbehandlung in einer
höheren Schicht erfolgt. Eine Ausnahme wäre der Fall, wenn ein
Fehler im sensiblen Datenbereich existiert. Das Prüfsummenlängenfeld
könnte für Verbindungen, welche partielle Fehlererkennung ermöglichen, dafür benutzt werden,
niedrigeren Schichten mitzuteilen,
in welchen Bereichen Fehler auftreten dürfen. Da der sensible
Teil des UDP-Lite-Pakets zwischen dem ersten Oktet des IP-Headers
und dem letzten Oktett liegt, welcher vom Prüfsummenlängenfeld
bekannt gegeben wird, kann der sensible Teil in der gleichen Weise
behandelt werden, wie ein UDP-Paket.\\

Da Verbindungen, welche keine partielle Fehlererkennung ermöglichen,
in einem Fehlerfall das Paket verwerfen müssen, wird das UDP-Lite-Paket
in gleicher Weise wie ein UDP-Paket behandelt.\\

Somit lässt sich bei UDP-Lite sagen, dass dieses Protokoll nur eine
Verbesserung erwirken kann, wenn in Schicht 2 des OSI-Referenzmodells
die Partielle Prüfsumme und die Prüfsummenlänge von UDP-Lite genutzt wird.
Dies würde seine Wirkung jedoch erst in Fehleranfälligen
Umgebungen entfalten.\\


\subsection{Kompatibilität mit UDP}

Da UDP und UDP-Lite eine ähnliche Syntax und Semantik hat,
können Anwendungen UDP-Lite anstatt UDP mit der
Eigenschaft nutzen, das ganze Paket mit einer Fehlerkorrektur zu
sichern. Des Weiteren sind durch die Ähnlichkeit nur geringe
Änderungen an Anwendungen vorzunehmen, um UDP-Lite verwenden zu können.\\

UDP-Lite hat eine eigene IP-Protokoll-Identifikation (136),
welche es einem Empfänger erlaubt zu erkennen, ob UDP-Lite oder UDP genutzt wird.
Ein Empfänger, welcher das UDP-Lite-Protokoll nicht kennt, wird ein
ICMP-Paket mit einer Fehlernachricht zurück senden.
Damit kann festgestellt werden, ob anderen Systemen UDP-Lite bekannt ist.
Diese Nachrichten können folgende sein:
\begin{itemize}
\item ICMP "Protocol Unreachable"
\item ICMPv6 "Payload Type Unknown"
\end{itemize}

Ein Problem würde bei der Verwendung einer gleichen UDP-Identifikation
entstehen, da ein UDP-Lite-Payload mit einer partiellen Prüfsumme
von UDP-Anwendungen verworfen wird und UDP-Pakete, welche nur teilweise
den IP-Payload füllen, nicht an UDP-Lite-Anwendungen weitergeleitet werden können.
Das Problem dabei wäre die fehlende Benachrichtigung an den Sender,
welches durch folgende Maßnahmen laut dem RFC eingedämmt werden könnte:
\begin{itemize}
\item Explizite Nutzung der Signalisierung innerhalb des Payloads
		ohne die partielle Prüfsumme zu verwenden, um dem Sender das
		Erkennen der UDP-Lite-Unterstützung zu ermöglichen
\item Nutzung eines anderen Protokolls zur Signalisierung,
		wie zum Beispiel SIP, damit erkannt werden kann,
		ob der Empfänger UDP-Lite nutzen kann
\end{itemize}

Da jedoch UDP-Lite eine eigene Identifikation besitzt,
müssen diese Varianten nicht verwendet werden.


\subsection{Sicherheitsbetrachtungen}

Der Sicherhitsaspekt von UDP-Lite hängt von der Interaktion mit
Authentifizierungs- und Verschlüsselungsverfahren ab. Da der sensible
Teil des Payloads beim Übertragen über verschiedene Netzsegmente variieren
kann, würde im Fall einer Beschädigung des Pakets
die Authentifizierung oder Verschlüsselung korrumpiert werden.
Dies gilt besonders wenn IPv6 verwendet wird, da hier eine Fehlerkorrektur
im IP-Paket vorhanden sein muss. Sollte ebenfalls IPSec mit ESP genutzt werden,
kann eine Verbindung nicht feststellen welches Transportprotokoll
verwendet wird. UDP-Lite kann in diesem Fall keinen Vorteil für
darüber liegende Codecs oder Anwendungen erwirken.\\

Stattdessen kann eine Verschlüsselung in der Transportschicht oder auf
Anwendungsebene angewandt werden, jedoch haben Verschlüsselungsverfahren,
(besonders Block-basierte) die Eigenheit, dass ein fehlerhaftes Bit
den gesamten Block unbrauchbar macht. Deshalb sollten spezielle
Stromchiffren verwendet werden, die dieses Verhalten minimieren.
Der Nachteil dieser ist jedoch, dass ein Angreifer möglicherweise
vorhersagbare Manipulationen am Payload vornehmen kann, ohne den
verschlüsselten Payload vollständig entschlüsseln zu müssen.\\



\subsection{IANA Protokollnummer}

Die IANA vergab UDP-Lite die Protokollnummer 136, änderte jedoch den
Namen auf UDPLite, da somit eine größere Bandbreite an Plattformen
mit diesem bedient werden kann, besonders diejenigen, die ein Minus-Zeichen
nicht unterstützen.









%\texttt{{\char'134}section}
%\footnote{This is the second footnote.  It
%starts a series of three footnotes that add nothing
%informational, but just give an idea of how footnotes work
%and look. It is a wordy one, just so you see
%how a longish one plays out.}
%\textbf{document} 



\section{Schlussfolgerung}

Abschließend ist zu sagen, dass UDP-Lite zwar Vorteile gegenüber UDP bietet,
diese jedoch nur bei fehleranfälligen Verbindungen Ihre Wirkung zeigen.
UDP-Lite hat zudem einen etwas höheren Einrichtungsaufwand,
dennoch im Fehlerfall einen ungleich großen Nutzen.
Deshalb sollte UDP-Lite für vorhersagbar schlechte Verbindungen verwendet werden,
muss jedoch bei bestehenden Systemen nicht notwendigerweise zum Einsatz kommen.
Bei kryptografischen Anwendungen hat UDP-Lite wenig Vorteile, weshalb
UDP für diese gleich oder besser geeignet scheint.

%\end{document}  % This is where a 'short' article might terminate

% The following two commands are all you need in the
% initial runs of your .tex file to
% produce the bibliography for the citations in your paper.
\bibliographystyle{abbrv}
\bibliography{template}  % template.bib is the name of the Bibliography in this case
% You must have a proper ".bib" file
%  and remember to run:
% latex bibtex latex latex
% to resolve all references
%
% ACM needs 'a single self-contained file'!
%
%APPENDICES are optional
%\balancecolumns
%\appendix
%Appendix A

\end{document}
